\chapter{Conclusion}

%summary - what, why, and how 
This thesis introduces the idea of Dynamic Wearable Technology. This concept envisions wearable devices as small autonomous robots that can move on and around the human body.
%we learned about climbing the fabric are. we experimented. 

%making the robots
Over the years this work has been interesting experience of experimentation, trails and errors. It was exciting to start with the blank slate, and try everything. When we had the initial vision of small wearable robots, we had no idea how to implememnt them, as I had no robotics experience. We experimented an toyed with various mechanisms such as using wheels and grooves. With the help of Stanford collaborators, we were able to make Rovables. The magnetic wheel mechanism worked surpisingly well. After working on the Rovables, the next logical challange was to make skin crawling robots. The Rovables did not provide direct access to the skin, and I thought there were interesting sensing oportuneties there. Climbing on the skin proved to be a challanging problem. Initially, I tried using adhesives, wheels, tracks, and various other standard mechanism. With trial and error, I no matter how you design it wheeled mechanisms don't work with such complex terrain. The nature confirms it, as there are no wheeled organisms. With wheels you just have to spin them and motion is created; assuming a flat terrain. A legged robot is more challanging, but provides greater control over the movement and terrain. Legs are delicate, and need tight control, with feedback, microcontroller and sensors. Once the climbing robot was operational, the next challange was finding the right applications. My 

There are some meta lessons that I learned in developing this work: 

1)Robotics are not particularly advanced at centemeter scale. At this scale there is not much market demand or research applications. It is hard to get good off-the-shelf parts. Most of the advaces in robotics are coming 
from the micro-world (MEMs, microfbrication) or large (e.g. drones, self-driving cars, industrial robots). The centemeter scale robots are mostly concerned with hobbyists such remove controlled cars, or medical applications such as insulin pumps or pressure cuffs.  Also, centermeter sized robots are akward for the software, as they can't run powerful operating systems or machine learning algorithms. 

2) Robotic autonomous operation is extremely hard. Developing hardware and electronics have a certain end goal, and by using simple physics one can figure out the limits of operation. Things such as speed and weight. Working on the autonomous operation and localization does not have a precise goal, and there are always imporovements that can be made and there are infinite real-world scenarios. A designer of autonomous system can't account for everything. The recent advances in machine learning have not been much help in this interprise. The size of the robot does not allow for advanced sensors and algorithms. 

Doing a thesis on robotics, makes it obliged to comment on the current state of Artificial Intellegence (AI). Hopefully, in the future, machine learning could run on small energy constrained systems. THere are indications that it will happen as there is great interest in real-time on-board image recognition, using popular neural networks (e.g. Convolutional neural nets). This approach has been collectively refered to as "Artificial Intelligence" now. Intel released the Neural Compute USB stickhip,  Google made a small "AI" chip and NVidia has been working on a small embedded GPU. I believe that that the more involved application of AI will be on analyzing the data robots are collecting. 

3) Feedback sensors are crutial. Many hardware issues could be outsourced to sensors and computers and the hardware just has to be good enough. Foe example, the robot can crawl on the skin using innexpensive hobby servo motors. By providing information from the vacuum pressure sensors, the motors can adjust themselves for correct skin attachment. 

4) Robotics are hard to justify. Every time, we demoed the robot projects, people asked why use it over the current technologies. I found it very difficult to answer that question. Almost anything a robot can do, a human can do better. In popular culture and science fiction has always presented a view of robots as powerful, scary and smart.   Had a different idea of robotics. 


%The thesis demonsrated two compelmentary approaches. First, Rovables, a clothing climbing robot, that pinches fabric with magnetic rolers. Second, Epidermal robots that use suction to attach to the skin. The robots contain on-board navigation that uses inertial measurement units and motor encoders to estimate position. 

%Dynamic wearable technology has applications in many areas: medicine, human-computer interactions, fashion and art. 

